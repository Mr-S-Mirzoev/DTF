\documentclass{article}
\usepackage{graphicx} % Required for inserting images
\usepackage[left=3cm, right=3cm, bottom=3cm, top=3cm]{geometry}	
\usepackage{setspace}
\onehalfspacing
\usepackage{float}
\usepackage{amsmath}

\title{Markowitz Portfolio Optimization with Inflation}
\author{Pablo Duce Cabeza, Djordje Nikolic, Sergey Mirzoev, Marc Tschudi}
\date{October 2023}

\begin{document}

\maketitle

\begin{figure}[h]
    \centering
    \includegraphics[width=0.4\textwidth]{figure/uzh_logo_e_pos.eps}
    \label{fig:mesh0}
\end{figure}

\newpage
\tableofcontents
\newpage
\listoffigures
\listoftables
\newpage

\section{Introduction}


In an era marked by economic uncertainties and dynamic financial landscapes, the importance of constructing resilient investment portfolios cannot be underestimated. Investors face a constant challenge in preserving and enhancing the real value of their wealth, especially when confronted with the persistent threat of inflation. As inflation erodes purchasing power, traditional investment strategies may prove insufficient, necessitating a paradigm shift towards a more sophisticated approach.

This paper delves into the realm of portfolio optimization using the renowned Markowitz framework, with a specific focus on navigating the complexities of an inflationary environment. Rather than adhering to conventional wisdom, our methodology begins with a comprehensive analysis of various asset classes to discern their correlation with inflation. Recognizing that not all assets respond uniformly to inflationary pressures, we aim to identify and select those that not only weather the storm but offer a shield against the corrosive effects of rising prices.

The initial phase of our investigation involves an examination of different asset classes, ranging from equities and fixed income securities to commodities and real assets. By scrutinizing historical data and employing statistical techniques, we seek to unveil the intricate relationships between these asset classes and inflation. Our objective is to discern patterns and identify the investment instruments that exhibit a robust correlation with inflationary trends.

Building on this foundation, we move forward to the application of the Markowitz portfolio optimization model. Recognizing that optimal portfolios are not one-size-fits-all, we tailor our approach to incofixed-incomeunique characteristics of inflation-sensitive assets. Through a judicious combination of these assets, we aim to construct portfolios that not only maximize returns but, crucially, provide a hedge against the eroding effects of inflation.

As a result Markowitz'skowitz portfolio optimization under the specter of inflation, we show that a portfolio can be constructed which lies on the efficient frontier compared to an equally weighted portfolio. By strategically selecting assets that exhibit resilience in the face of inflation, the optimized portfolio outperforms an the CPI adjusted equally weighted portfolio substantially.

\newpage


\section{Asset Classes}

For an asset class to offer protection against inflation, the asset returns have to be positively correlated with the CPI (Consumer Price Index) when we assume that only long positions can be held; so, when the CPI increases, the asset returns should also increase. The approach we present will be based on filtering. Starting from a large pool of potential assets that could hedge against inflation, we want to understand what are the assets that are more prone to have a positive relationship with inflation. That means analyzing the yearly inflation rates, and see what are the subsequent returns of a given class. We therefore expect that besides high correlation with inflation, when inflation is very high, returns also should be very high (tail inflation corresponds with the tail of returns). We test the following asset classes and sectors for a hedge against inflation:

\begin{itemize}
    \item XHB: Median Sale Price of Houses Sold in the United States
    \item CPIAUCSL: Consumer Price Index for All Urban Consumers
    \item DCOILWTICO: Crude Oil Prices: West Texas Intermediate (WTI)
    \item GSPC: S\&P 500 Index
    \item GC=F: Gold Futures
    \item GS10: 10-Year Treasury Constant Maturity Rate
    \item GS30: 30-Year Treasury Constant Maturity Rate
    \item VNQ: REITs (Vanguard Real Estate ETF)
    \item DBC: Commodities (Invesco DB Commodity Index Tracking Fund)
    \item FXE: Foreign Currencies (Currency Shares Euro Trust)
    \item BTC-USD: Cryptocurrencies (Bitcoin)
    \item IGF: Infrastructure Funds (iShares Global Infrastructure ETF)
    \item XLE: Energy Stocks (Energy Select Sector SPDR Fund)
\end{itemize}



\end{document}
